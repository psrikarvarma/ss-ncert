\documentclass{article}

\title{Analog Assignment}

\begin{document}

\section*{Analog Assignment}
\textit{ee23btech11215,Penmetsa Srikar Varma,IC Design and Technology}
\subsection*{Question 23 from chapter 15:WAVES of class 11}
\textbf{Q23)}  A narrow sound pulse (for example, a short pip by a whistle) is sent across a
medium. (a) Does the pulse have a definite (i) frequency, (ii) wavelength, (iii) speed
of propagation? (b) If the pulse rate is 1 after every 20 s, (that is the whistle is
blown for a split of second after every 20 s), is the frequency of the note produced
by the whistle equal to 1/20 or 0.05 Hz ?\\
\\ \textbf{Answer:}\\\\
\textbf{a)} Let us assume that sound pulse produced in a medium of gas having a specific molecular weight \textbf{M} and having adiabatic constant $\gamma$ which is at a constant temperature \textbf{T}\\
Then velocity \textbf{V} of sound pulse is given by:\\
$$\textbf{V}=\sqrt{\left(\frac{\gamma RT}{M}\right)}$$ (where \textbf{R} is Universal gas Constant)\\
\\Hence from the above formula the velocity \textbf{V} of sound wave remains constant but not frequency \textbf{$\nu$} and wavelength \textbf{$\lambda$}\\\\
\textbf{b)} We know that for a sound pulse travelling in a medium\\\\
The general equation of a point on the wave is given by:\\
$$\textit{y=A sin(kx+$\omega$t)}$$(where \textit{x},\textit{y} are co-ordinates corresponding to point under observation\\
\textit{A} = Ampiltude of sound wave\\ 
\textit{k} = Wave number\\
$\omega$ = Angular frequency of the wave\\
\textit{t} = time
)\\\\
And we know the relation that:
$$\nu = \frac{\omega}{2\pi}$$
Hence, The frequency of the note $\nu$ produced will not be equal to 0.05 Hz or $\frac{1}{20}$ Hz 

\end{document}
