% \iffalse
\let\negmedspace\undefined
\let\negthickspace\undefined
\documentclass[beamer]{IEEEtran}
\usepackage{cite}
\usepackage{amsmath,amssymb,amsfonts,amsthm}
\usepackage{algorithmic}
\usepackage{graphicx}
\usepackage{textcomp}
\usepackage{xcolor}
\usepackage{txfonts}
\usepackage{listings}
\usepackage{enumitem}
\usepackage{mathtools}
\usepackage{gensymb}
\usepackage{comment}
\usepackage[breaklinks=true]{hyperref}
\usepackage{tkz-euclide} 
\usepackage{listings}
\usepackage{gvv}                                        
\def\inputGnumericTable{}                                 
\usepackage[latin1]{inputenc}                                
\usepackage{color}                                            
\usepackage{array}                                            
\usepackage{longtable}                                       
\usepackage{calc}                                             
\usepackage{multirow}                                         
\usepackage{hhline}                                           
\usepackage{ifthen}                                           
\usepackage{lscape}
\usepackage[export]{adjustbox}

\newtheorem{theorem}{Theorem}[section]
\newtheorem{problem}{Problem}
\newtheorem{proposition}{Proposition}[section]
\newtheorem{lemma}{Lemma}[section]
\newtheorem{corollary}[theorem]{Corollary}
\newtheorem{example}{Example}[section]
\newtheorem{definition}[problem]{Definition}
\newcommand{\BEQA}{\begin{eqnarray}}
\newcommand{\EEQA}{\end{eqnarray}}
\newcommand{\define}{\stackrel{\triangle}{=}}
\theoremstyle{remark}
\newtheorem{rem}{Remark}
\begin{document}
\parindent 0px
\bibliographystyle{IEEEtran}

\title{Assignment\\[1ex]11.15-23}
\author{ee23btech11215 - Penmetsa Srikar Varma$^{}$% <-this % stops a space
}
\maketitle
\newpage
\bigskip

\renewcommand{\thefigure}{\theenumi}
\renewcommand{\thetable}{\theenumi}
\section*{Question:}
Q23) A narrow sound pulse (for example, a short pip by a whistle) is sent across a
medium.\\ \brak{a} Does the pulse have a definite \brak{i} frequency, \brak{ii} wavelength, \brak{iii} speed
of propagation?\\ \brak{b} If the pulse rate is 1 after every 20 s, (that is the whistle is
blown for a split of second after every 20 s), is the frequency of note produced
by whistle equal to 1/20 or 0.05 Hz ?
\section*{Solution:}
{\centering
Table of Parameters\\
}
\begin{table}[h]
    \centering
    \begin{tabular}{|c|c|}
        \hline
         Parameter & Name of Parameter  \\
        \hline
         M & Molecular Weight of gas\\
         \hline
         $\gamma$ & Adiabatic Constant of gas\\
         \hline
         T & Temperature of gas \\
         \hline
         V & velocity of gas \\
         \hline
         R & Universal Gas Constant \\
         \hline
         $\nu$ & Frequency of Sound wave \\
         \hline
         $\lambda$ & Wavelength of Sound wave\\
         \hline
         A & Amplitude of Sound wave\\
         \hline
         x,y & Co-ordinates of point on wave\\
         \hline
         k & wave number\\
         \hline
         $\omega$ & Angular Frequency of wave\\
         \hline
         t & time\\
         \hline
    \end{tabular}
    \label{tab:t1}
\end{table}

\brak{a} Let us assume that sound pulse produced in a medium of gas having a specific molecular weight M and having adiabatic constant $\gamma$ which is at a constant temperature T\\
Then velocity V of sound pulse is given by:\\
\begin{align}
    \label{a1}
    V=\sqrt{\left(\frac{\gamma RT}{M}\right)}
\end{align}(where R is Universal gas Constant)\\

Hence from \brak{\ref{a1}} the velocity V of sound wave remains constant but not frequency $\nu$ and wavelength $\lambda$\\\\
\brak{b} We know that for a sound pulse travelling in a medium\\\\
The general equation of a point on the wave is given by:\\
\begin{align}
    \label{a2}
     y=A.sin\brak{kx-\omega t}
\end{align}

And we know the relation that:
\begin{align}
    \label{a3}
    \nu = \frac{\omega}{2\pi}
\end{align}
Hence, The frequency of the note $\nu$ produced will not be equal to 0.05 Hz or $\frac{1}{20}$ Hz 

\end{document}
