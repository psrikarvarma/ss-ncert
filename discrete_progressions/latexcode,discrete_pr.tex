% \iffalse
\let\negmedspace\undefined
\let\negthickspace\undefined
\documentclass[beamer]{IEEEtran}
\usepackage{cite}
\usepackage{amsmath,amssymb,amsfonts,amsthm}
\usepackage{algorithmic}
\usepackage{graphicx}
\usepackage{textcomp}
\usepackage{xcolor}
\usepackage{txfonts}
\usepackage{listings}
\usepackage{enumitem}
\usepackage{mathtools}
\usepackage{gensymb}
\usepackage{comment}
\usepackage[breaklinks=true]{hyperref}
\usepackage{tkz-euclide} 
\usepackage{listings}
\usepackage{gvv}                                        
\def\inputGnumericTable{}                                 
\usepackage[latin1]{inputenc}                                
\usepackage{color}                                            
\usepackage{array}                                            
\usepackage{longtable}                                       
\usepackage{calc}                                             
\usepackage{multirow}                                         
\usepackage{hhline}                                           
\usepackage{ifthen}                                           
\usepackage{lscape}
\usepackage[export]{adjustbox}

\newtheorem{theorem}{Theorem}[section]
\newtheorem{problem}{Problem}
\newtheorem{proposition}{Proposition}[section]
\newtheorem{lemma}{Lemma}[section]
\newtheorem{corollary}[theorem]{Corollary}
\newtheorem{example}{Example}[section]
\newtheorem{definition}[problem]{Definition}
\newcommand{\BEQA}{\begin{eqnarray}}
\newcommand{\EEQA}{\end{eqnarray}}
\newcommand{\define}{\stackrel{\triangle}{=}}
\theoremstyle{remark}
\newtheorem{rem}{Remark}
\begin{document}
\parindent 0px
\bibliographystyle{IEEEtran}

\title{Assignment\\[1ex]10.5.4-2}
\author{ee23btech11215 - Penmetsa Srikar Varma$^{}$% <-this % stops a space
}
\maketitle
\newpage
\bigskip

\renewcommand{\thefigure}{\theenumi}
\renewcommand{\thetable}{\theenumi}
\section*{Question:}
Q2) The sum of the third and the seventh terms of AP is 6 and their product is 8. Find the sum of first sixteen terms of the AP\\
\section*{Solution:}
{
\centering
Table of Parameters\\
}
\begin{table}[h]
    \centering
    \documentclass{article}
\begin{document}
\textbf{Input Table:}\\
\begin{center}
\begin{tabular}{|c|c|}  
\hline
     Input Variables & Input Condition \\
\hline
     $a(2)$ & third term of \textit{AP}\\
\hline
     $a(6)$ & seventh term of \textit{AP}\\
\hline
     $a(2)$+$a(6)$ & 6 \\
\hline
     $a(2).a(6)$ & 8 \\
\hline
     $S(16)$ & sum of first 16 terms of \textit{AP}\\
\hline
\end{tabular}
\end{center}
\end{document}

    \label{table of parameters}
\end{table}

Then general term y\brak{n} of arithmetic progression is given by:
\begin{align}
\label{q1}
y_{1,2}\brak{n}&=y_{1,2}\brak{0}+n\ d_{1,2}
\end{align}
 
Then from table of parameters,
\begin{align}
y_{1,2}\brak{6}\brak{6-y_{1,2}\brak{6}}=8
\end{align}
\begin{align}
\label{q7}
    y_{1,2}\brak{6}&=2\ \text{or}\ 4
\end{align}
Then from table and $\brak{\ref{q7}}$
\begin{align}
\label{q8}
    y_{1,2}\brak{2}&=4\ \text{or}\ 2
\end{align}
for $y_1\brak{2}$ = 2 and $y_1\brak{6}$ = 4
\begin{align}
y_1\brak{0}&=1,\ d_1=\frac{1}{2}
\end{align}
for $y_2\brak{2}$ = 4 and $y_2\brak{6}$ = 2
\begin{align}
y_2\brak{0}&=1,\ d_2=-\frac{1}{2}
\end{align}
We know that the sum of first n terms of arithmetic progression is given by:
\begin{align}
\label{q9}
S_1\brak{n}&= \frac{n}{2}\brak{2y_1\brak{0}+\brak{n-1}d_1}u\brak{n}
\end{align}
Then from \brak{\ref{q9}}
\begin{align}
\label{q10}
S_1\brak{16}&= \frac{16}{2}\brak{2y_1\brak{0}+15d_1}
\end{align}
Hence from \brak{\ref{q10}},
for $y_1\brak{0}=1$,\ $d_1=\frac{1}{2}$
\begin{align}
\label{11}
S_1\brak{16}&=76
\end{align}
or from \brak{\ref{q10}},
for $y_1\brak{0}$=5,\ $d_2=-\frac{1}{2}$
\begin{align}
\label{12}
S_2\brak{16}&=20
\end{align}
The general term of AP $y_{1,2}\brak{n}$ and sum of first n terms of AP $S_{1,2}\brak{n}$ are given by:
\begin{align}
y_1\brak{n}&=\brak{\frac{n+2}{2}}u\brak{n}\ \text{and}\ S_1\brak{n}=\brak{\frac{n\brak{n+3}}{4}}u\brak{n}
\end{align}
\begin{align}
y_2\brak{n}&=\brak{\frac{10-n}{2}}u\brak{n}\ \text{and}\ S_2\brak{n}=\brak{\frac{n\brak{21-n}}{4}}u\brak{n}
\end{align}

\begin{figure}[h]
    \centering
    \includegraphics[scale=0.60]{py_3.png}
    \label{fig:x1n}
\end{figure}

\begin{center}
    Graph of $y_1\brak{n}$
\end{center}

$z$-Transform of y\brak{n} is given from table :

\begin{align}
Y_1\brak{z}&=\sum_{k=0}^{\infty} \brak{\frac{k+2}{2}}\ z^{-k}
\end{align}

\begin{align}
    \label{q13}
    Y_1\brak{z}&=\frac{1}{2}\brak{\sum_{k=0}^{\infty}k\ z^{-k}}+\sum_{k=0}^{\infty}z^{-k}
\end{align}
\begin{align}
    \label{q13}
    Y_1\brak{z}&=\frac{1}{2}\brak{\frac{z^{-1}}{\brak{1-z^{-1}}^2}}+\frac{1}{1-z^{-1}}
\end{align}
So,
\begin{align} Y_1\brak{z}&=\frac{2-z^{-1}}{2.\brak{1-z^{-1}}^2},\quad |z^{-1}|<1\end{align}

\begin{figure}[h]
    \centering
    \includegraphics[scale=0.60]{py_5.png}
    \label{fig:x2n}
\end{figure}

\begin{center}
    Graph of $y_2\brak{n} $
\end{center}

Similarly,
\begin{align}Y_2\brak{z}&=\sum_{k=0}^{\infty} \brak{\frac{10-k}{2}}\ z^{-k}\end{align}
\begin{align}Y_2\brak{z}&=5\brak{\sum_{k=0}^{\infty}z^{-k}}-\frac{1}{2}\brak{\sum_{k=0}^{\infty}k\ z^{-k}}\end{align}

So,
\begin{align}Y_2\brak{z}&=\frac{10-11z^{-1}}{2.\brak{1-z^{-1}}^2},\quad |z^{-1}|<1 \end{align}

\begin{figure}[h]
    \centering
    \includegraphics[scale=0.60]{py_4.png}
    \label{fig:s1n}
\end{figure}

\begin{center}
    Graph of $S_1\brak{n}$
\end{center}

Similarly for sum of first n terms of AP,

\begin{align}
S\brak{z}&=\sum_{k=-\infty}^{\infty}\left(\brak{y\brak{0}-\frac{d}{2}}k+\frac{d}{2}\ k^2\right)\ u\brak{k}z^{-k}
\end{align}

\begin{align}
\label{q17}
    S\brak{z}&=\brak{y\brak{0}-\frac{d}{2}}\left(\sum_{k=0}^{\infty}k\ z^{-k}\right)+\frac{d}{2}\left(\sum_{k=0}^{\infty}k^2\ z^{-k}\right)
\end{align}

\begin{align}
\label{w1}
S\brak{z}&=y\brak{0}\brak{\frac{z}{\brak{z-1}^2}}+d\brak{\frac{z}{\brak{z-1}^3}}
\end{align}

for $y_1\brak{0}=1$ and $d_1=\frac{1}{2}$

\begin{align}
S_1\brak{z}&=\frac{z\brak{z-\frac{1}{2}}}{\brak{z-1}^3},\quad |z|>1
\end{align}



for $y_2\brak{0}=5$ and $d_2=-\frac{1}{2}$

\begin{align}S_2\brak{z}&=\frac{z\brak{5z-\frac{11}{2}}}{\brak{z-1}^3},\quad |z|>1\end{align}

\begin{figure}[h]
    \centering
    \includegraphics[scale=0.60]{py_6.png}
    \label{s2n}
\end{figure}

\begin{center}
Graph of $S_2\brak{n}$
\end{center}

Inverse $z$-transform of $S_{1,2}\brak{z}$ by counter integral method is given by:

\begin{align}
s_{1,2}\brak{n}&=\oint_C S_{1,2}(z)\ z^{n-1}\, dz
\end{align}

from \brak{\ref{w1}},

\begin{align}
    s_{1,2}\brak{n}&=y_{1,2}\brak{0}\oint_C \brak{\frac{z}{\brak{z-1}^2}dz} + d_{1,2}\brak{\oint_C  \frac{z}{\brak{z-1}^3} \, dz}
\end{align}

We can observe that pole is repeated 2,3 times so,\\m=2 and 3 respectively:
\begin{align}
    s_{1,2}\brak{n}&=y_{1,2}\brak{0}\left|\frac{1}{1!}\frac{d}{dz}\brak{z^n}\right|_{z=1} + d_{1,2}\left|\frac{1}{2!}\frac{d^2}{dz^2}\brak{z^n}\right|_{z=1}
\end{align}

\begin{align}
    s_{1,2}\brak{n}&=\brak{n\ y_{1,2}\brak{0} + n\brak{n-1}\frac{d_{1,2}}{2}}u\brak{n}
\end{align}
\begin{align}
\label{39}
    s_1\brak{n}&=\frac{n\brak{n+3}}{4}u\brak{n},\quad s_1\brak{16}=76
\end{align}
and,
\begin{align}
\label{40}
    s_2\brak{n}&=\frac{n\brak{21-n}}{4}u\brak{n},\quad s_2\brak{16}=20
\end{align}

we can observe that \brak{\ref{11}} and \brak{\ref{39}},\brak{\ref{12}} and \brak{\ref{40}} are the same results
\end{document}
