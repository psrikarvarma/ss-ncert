% \iffalse
\let\negmedspace\undefined
\let\negthickspace\undefined
\documentclass[beamer]{IEEEtran}
\usepackage{cite}
\usepackage{amsmath,amssymb,amsfonts,amsthm}
\usepackage{algorithmic}
\usepackage{graphicx}
\usepackage{textcomp}
\usepackage{xcolor}
\usepackage{txfonts}
\usepackage{listings}
\usepackage{enumitem}
\usepackage{mathtools}
\usepackage{gensymb}
\usepackage{comment}
\usepackage[breaklinks=true]{hyperref}
\usepackage{tkz-euclide} 
\usepackage{listings}
\usepackage{gvv}                                        
\def\inputGnumericTable{}                                 
\usepackage[latin1]{inputenc}                                
\usepackage{color}                                            
\usepackage{array}                                            
\usepackage{longtable}                                       
\usepackage{calc}                                             
\usepackage{multirow}                                         
\usepackage{hhline}                                           
\usepackage{ifthen}                                           
\usepackage{lscape}
\usepackage[export]{adjustbox}

\newtheorem{theorem}{Theorem}[section]
\newtheorem{problem}{Problem}
\newtheorem{proposition}{Proposition}[section]
\newtheorem{lemma}{Lemma}[section]
\newtheorem{corollary}[theorem]{Corollary}
\newtheorem{example}{Example}[section]
\newtheorem{definition}[problem]{Definition}
\newcommand{\BEQA}{\begin{eqnarray}}
\newcommand{\EEQA}{\end{eqnarray}}
\newcommand{\define}{\stackrel{\triangle}{=}}
\theoremstyle{remark}
\newtheorem{rem}{Remark}
\begin{document}
\parindent 0px
\bibliographystyle{IEEEtran}

\title{Assignment\\[1ex]10.5.4-2}
\author{ee23btech11215 - Penmetsa Srikar Varma$^{}$% <-this % stops a space
}
\maketitle
\newpage
\bigskip

\renewcommand{\thefigure}{\theenumi}
\renewcommand{\thetable}{\theenumi}
\section*{Question:}
Q2) The sum of the third and the seventh terms of AP is 6 and their product is 8. Find the sum of first sixteen terms of the AP\\
\section*{Solution:}
{
\centering
Table of Parameters\\
}
\begin{table}[h]
    \centering
    \begin{tabular}{|c|c|}
    \hline
     Input Variables & Input Condition \\
\hline
     x\brak{2}+x\brak{6}& 6 \\
\hline
     x\brak{2}.x\brak{6} & 8 \\
\hline
     $x_i$\brak{n} &  general term of $\text{i}^\text{th}$ AP sequence\\
\hline
     $y_i$\brak{n} &  sum of first n terms of $\text{i}^\text{th}$ AP sequence\\
\hline
     $x_i$\brak{0} & first term of $\text{i}^\text{th}$ AP sequence\\
\hline
     $d_i$ & common difference of $\text{i}^\text{th}$ AP sequence\\
\hline
     $x_i\brak{n}\system{Z}X_i\brak{z}$ & $z$-transform of $x_i\brak{n}$ \\
\hline
     $y_i\brak{n}\system{Z}Y_i\brak{z}$ & $z$-transform of $y_i\brak{n}$ \\
\hline
    \end{tabular}
    \label{table of parameters}
\end{table}

Then general term x\brak{\text{n}} of arithmetic progression is given by:
\begin{align}
\label{q1}
x_i\brak{n}&=x_i\brak{0}+n\ d_i
\end{align}
 
Then from table of parameters,
\begin{align}
x_i\brak{6}\brak{6-x_i\brak{6}}=8
\end{align}
\begin{align}
\label{q7}
    x_1\brak{6}&=4\ \text{and}\ x_2\brak{6}=2
\end{align}
Then from table and $\brak{\ref{q7}}$
\begin{align}
\label{q8}
    x_1\brak{2}&=2\ \text{and}\ x_2\brak{2}=4
\end{align}
for $x_1\brak{2}$ = 2 and $x_1\brak{6}$ = 4
\begin{align}
x_1\brak{0}&=1,\ d_1=\frac{1}{2}
\end{align}
for $x_2\brak{2}$ = 4 and $x_2\brak{6}$ = 2
\begin{align}
x_2\brak{0}&=1,\ d_2=-\frac{1}{2}
\end{align}
We know that the sum of first n terms of arithmetic progression is given by:
\begin{align}
\label{q9}
y_i\brak{n}&= \frac{n}{2}\brak{2x_i\brak{0}+\brak{n-1}d_i}u\brak{n}
\end{align}
\begin{align}
\label{11}
y_1\brak{16}&=76
\end{align}
\begin{align}
\label{12}
y_2\brak{16}&=20
\end{align}
The general term of AP $x_i\brak{n}$ and sum of first n terms of AP $y_i\brak{n}$ are given by:\\
\begin{align}
x_1\brak{n}&=\brak{\frac{n+2}{2}}u\brak{n}\ \text{and}\ y_1\brak{n}=\brak{\frac{n\brak{n+3}}{4}}u\brak{n}
\end{align}
\begin{align}
x_2\brak{n}&=\brak{\frac{10-n}{2}}u\brak{n}\ \text{and}\ y_2\brak{n}=\brak{\frac{n\brak{21-n}}{4}}u\brak{n}
\end{align}\\
\begin{figure}[h]
    \centering
    \includegraphics[scale=0.60]{py_3.png}
    \label{fig:x1n}
\end{figure}
\begin{center}
    Graph of $x_1\brak{n}$\\
\end{center}
\begin{figure}[h]
    \centering
    \includegraphics[scale=0.60]{py_5.png}
    \label{fig:x2n}
\end{figure}
\begin{center}
    Graph of $x_2\brak{n}$\\
\end{center}
\begin{figure}[h]
    \centering
    \includegraphics[scale=0.60]{py_4.png}
    \label{fig:s1n}
\end{figure}
\begin{center}
    Graph of $y_1\brak{n}$\\[3ex]
\end{center}
\begin{figure}[h]
    \centering
    \includegraphics[scale=0.60]{py_6.png}
    \label{s2n}
\end{figure}
\begin{center}
Graph of $y_2\brak{n}$\\[5ex]
\end{center}

$z$-Transform of $x_1\brak{\text{n}}$ is given by:\\
\begin{align}
X_1\brak{z}&=\sum_{k=-\infty}^{\infty} x_1\brak{k}z^{-k}\\
\end{align}
\begin{align} X_1\brak{z}&=\frac{2-z^{-1}}{2.\brak{1-z^{-1}}^2},\quad |z^{-1}|<1
\end{align}

Similarly,
\begin{align}
X_1\brak{z}&=\sum_{k=-\infty}^{\infty} x_2\brak{k}z^{-k}\\
\end{align}
\begin{align}X_2\brak{z}&=\frac{10-11z^{-1}}{2.\brak{1-z^{-1}}^2},\quad |z^{-1}|<1 \end{align}\\[1ex]

Similarly for sum of first n terms of AP,

\begin{align}
Y_i\brak{z}&=\sum_{k=-\infty}^{\infty} y_i\brak{k}z^{-k}
\end{align}

\begin{align}
\label{w1}
Y_i\brak{z}&=x_i\brak{0}\brak{\frac{z}{\brak{z-1}^2}}+d_i\brak{\frac{z}{\brak{z-1}^3}}
\end{align}

\begin{align}
Y_1\brak{z}&=\frac{z\brak{z-\frac{1}{2}}}{\brak{z-1}^3},\quad |z|>1
\end{align}

\begin{align}Y_2\brak{z}&=\frac{z\brak{5z-\frac{11}{2}}}{\brak{z-1}^3},\quad |z|>1\end{align}\\

Inverse $z$-transform of $y_i\brak{z}$ by counter integral method is given by:

\begin{align}
y_i\brak{n}&=\oint_C Y_i(z)\ z^{n-1}\, dz
\end{align}

\begin{align}
    y_i\brak{n}&=x_i\brak{0}\oint_C \brak{\frac{z}{\brak{z-1}^2}dz} + d_i\brak{\oint_C  \frac{z}{\brak{z-1}^3} \, dz}
\end{align}\\

We can observe that pole is repeated 2,3 times so, m=2 and 3 respectively:\\
\begin{align}
    y_i\brak{n}&=x_i\brak{0}\left|\frac{1}{1!}\frac{d}{dz}\brak{z^n}\right|_{z=1} + d_i\left|\frac{1}{2!}\frac{d^2}{dz^2}\brak{z^n}\right|_{z=1}
\end{align}

\begin{align}
    y_i\brak{n}&=\brak{n\ x_i\brak{0} + n\brak{n-1}\frac{d_i}{2}}u\brak{n}
\end{align}
\begin{align}
\label{39}
    y_1\brak{n}&=\frac{n\brak{n+3}}{4}u\brak{n},\quad y_1\brak{16}=76
\end{align}

\begin{align}
\label{40}
    y_2\brak{n}&=\frac{n\brak{21-n}}{4}u\brak{n},\quad y_2\brak{16}=20
\end{align}

\end{document}
